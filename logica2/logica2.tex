\documentclass[10pt, headsepline, captions=tableabove, xcolor=table]{beamer}
% ==> \documentclass[10pt, handout]{beamer}    to print 4 in one page
\usepackage[utf8]{inputenc}
% \usepackage{multicol}
\usepackage{amsmath}
\usepackage[brazilian]{babel}
\usepackage{xcolor}
\usepackage{fancybox}
\usepackage{pgfpages}
\usepackage{pifont}
%\usepackage{enumitem}
% ==> \pgfpagesuselayout{4 on 1}[]      to print 4 in one page

\usetheme{Warsaw}
\usecolortheme{default}
\usefonttheme{professionalfonts}
\graphicspath{{./imgs/}}
\setbeamertemplate{caption}[numbered]
\setbeamertemplate{footline}[frame number]
\setbeamercovered{transparent}
\setbeamertemplate{section in toc}{\inserttocsectionnumber.~\inserttocsection} 
% \setbeamertemplate{navigation symbols}{} % remove navigation panel
% TOC style
%\setbeamercovered{invisible}
%\geometry{left=1cm, right=1cm, top=0.5cm, bottom=0.5cm}

\title{Lógica Matemática}
\subtitle{Parte 2}
\author[Paulo Pinheiro]
{Dr. Paulo Vinicius Pereira Pinheiro\inst{1}}
\institute[UNIFAP]
{
    \inst{1}
    Centro Universitário Paraíso do Ceará\\
    UNIFAP
}
%
\date{\small{Acesse estes slides em:\\ \url{https://github.com/paulovpp/slides}}\newline \\Última atualização:\\ \today}
\logo{\includegraphics[height=0.7cm]{imgs/UNIFAP.png}}
%
% % \makeatletter
% % A new section definition to automatically set labels with the name: sec:sectionnumber
% \let\oldsection\section
% \renewcommand{\section}[1]{\oldsection{#1}\label{sec:\thesection}}
% % A new subsection definition to automatically set labels with the name: sec:sectionnumber.subsectionnumer
% \let\oldsubsection\subsection
% \renewcommand{\subsection}[1]{\oldsubsection{#1}\label{sec:\thesection.\thesubsection}}
% %init some calculators
% \newcounter{calculator}
% \newcounter{calcmaxsec}
% \newcounter{calcmaxsubsec}
% \newcommand{\calcsubsection}[1]%procedure to get the name of section thissection+i
% {
%     \setcounter{calculator}{\thesubsection}\addtocounter{calculator}{#1}
%     \@ifundefined{r@sec:\thesection.\thecalculator}{}{\color{fg!40!bg}\nameref{sec:\thesection.\thecalculator}}
% }
% \newcommand{\calcsection}[1]%procedure to get the name of subsection thissection.thissubsection+i
% {
%     \setcounter{calculator}{\thesection}\addtocounter{calculator}{#1}
%     \@ifundefined{r@sec:\thesection}{}{\color{fg!40!bg}\nameref{sec:\thecalculator}}
% }
% \newcommand{\Testframe}%some testframe definition
%     {
%         \begin{frame}
%             The Section: \thesection\\
%             The Subsection: \thesubsection\\
%             The Calculator: \thecalculator\\
%             Count\thesection
%         \end{frame}
%     }
% %Defining the layout%%%%%%%%%%%%%%%%%%%%%%%%%%%
% \setbeamertemplate{headline}
% {%
%   \leavevmode%
%   \@tempdimb=2.4375ex%
%   \ifnum\beamer@subsectionmax<\beamer@sectionmax%
%      \multiply\@tempdimb by\beamer@sectionmax%
%   \else%
%      \multiply\@tempdimb by\beamer@subsectionmax%
%   \fi%
%   \ifdim\@tempdimb>0pt%
%      \advance\@tempdimb by 1.125ex%
%      \begin{beamercolorbox}[wd=.5\paperwidth,ht=\@tempdimb,right,rightskip=1em]{section in head/foot}%
%         \vbox to \@tempdimb{%
%         \setcounter{calcmaxsec}{\thesection}\addtocounter{calcmaxsec}{-\beamer@sectionmax}
%         \ifnum\thecalcmaxsec>-1\vfill{\calcsection{-3}}\fi%
%         \ifnum\thecalcmaxsec>-2\vfill{\calcsection{-2}}\fi%
%         \ifnum\thesection>1\vfill{\calcsection{-1}}\fi%
%         \ifnum\thesection>0\vfill\insertsectionhead\fi
%         \ifnum\thecalcmaxsec<0\vfill{\calcsection{1}}\fi
%         \ifnum\thecalcmaxsec<-1\vfill{\calcsection{2}}\fi
%         \ifnum\thesection<2\vfill{\calcsection{3}}\fi
%         \ifnum\thesection<1\vfill{\calcsection{4}}\fi
%         \vfill%
%      }%
%      \end{beamercolorbox}%
%      \begin{beamercolorbox}[wd=.5\paperwidth,ht=\@tempdimb]{subsection in head/foot}%
%         \vbox to \@tempdimb{%
%         \setcounter{calcmaxsubsec}{\thesubsection}\addtocounter{calcmaxsubsec}{-\beamer@subsectionmax}
%         \ifnum\thecalcmaxsubsec>-1 \vfill{\calcsubsection{-3}}\fi%
%         \ifnum\thecalcmaxsubsec>-2 \vfill{\calcsubsection{-2}}\fi%
%         \ifnum\thesubsection>1 \vfill{\calcsubsection{-1}}\fi%
%         \ifnum\thesubsection>0 \vfill\insertsubsectionhead\fi
%         \ifnum\beamer@subsectionmax>1{\ifnum\thecalcmaxsubsec<0\vfill{\calcsubsection{1}}\fi}\fi
%         \ifnum\beamer@subsectionmax>2{\ifnum\thecalcmaxsubsec<-1\vfill{\calcsubsection{2}}\fi}\fi
%         \ifnum\beamer@subsectionmax>3{\ifnum\thesubsection<2\vfill{\calcsubsection{3}}\fi}\fi
%         \ifnum\beamer@subsectionmax>4{\ifnum\thesubsection<1\vfill{\calcsubsection{4}}\fi}\fi
%         \vfill%
%       }%
%      \end{beamercolorbox}%
%   \fi%
% }%
% \makeatother %%%%%%%%%%%%%%%%%%%%%%

\usepackage{totcount}
\newcounter{totalsection}
\regtotcounter{totalsection}

\AtBeginDocument{%
  \pretocmd{\section}{\refstepcounter{totalsection}}{}{}%
}%

% number of subsections per section %%%%%%%%%%%%%%%%%%%%%%%%%%%%%%%%%%
\usepackage{xcntperchap}
\RegisterCounters{section}{subsection}
\newcounter{totalsubsection}
\setcounter{totalsubsection}{0}




\newcounter{currentsub}
\setcounter{currentsub}{0}
\newcounter{totsection}
\AtBeginSection[]{%
    \setcounter{currentsub}{\ObtainTrackedValueExp[\thesection]{section}{subsection}}
    \recalc
}

\makeatletter
\setbeamertemplate{headline}{%
  \leavevmode%
  \@tempdimb=2.4375ex%
  \ifnum\thecurrentsub<\beamer@sectionmax%
    \multiply\@tempdimb by\beamer@sectionmax%
  \else%
    \multiply\@tempdimb by\thecurrentsub%
  \fi%
  \ifdim\@tempdimb>0pt%
    \advance\@tempdimb by 1.825ex%
    \begin{beamercolorbox}[wd=.5\paperwidth,ht=\@tempdimb]{section in head/foot}%
      \vbox to\@tempdimb{\vfil\insertsectionnavigation{.5\paperwidth}\vfil}%
    \end{beamercolorbox}%
    \begin{beamercolorbox}[wd=.5\paperwidth,ht=\@tempdimb]{subsection in head/foot}%
      \vbox to\@tempdimb{\vfil\insertsubsectionnavigation{.5\paperwidth}\vfil}%
    \end{beamercolorbox}%
  \fi%
}

\newcommand{\recalc}{\beamer@calculateheadfoot}
\makeatother
%%%%%%%%%%%%%%%%%%%%%%%%%%%%%%%%%%%%%%%%%%%%%%%%%%%%%%%%%%%%%%%%%%%%%%%%%%%%%
%
\begin{document}
% \setcounter{tocdepth}{1} % Show sections
% \setcounter{tocdepth}{2} % + subsections
% Title page frame
\begin{frame}
    \titlepage
\end{frame}

% Remove logo from the next slides
\logo{}

% Outline frame
\begin{frame}[t]{Sumário}
    \tableofcontents[sections={1-3}]
\end{frame}
%
% \begin{frame}[t]{Sumário}
%     \tableofcontents[sections={4-}]
% \end{frame}

%% -------------------------------- %% -----------------------------------%%
%% Section 1 - Tabela verdade
\section{Tabela verdade}
%
\subsection{Definições iniciais}
%
\begin{frame}[c]
    \frametitle{Definições iniciais}
    \framesubtitle{Introductory definitions to the topic}
    \begin{block}{Número de linhas}
        O número de linhas de uma tabela verdade de uma proposição composta depende do número $n$ de proposições simples que a integram sendo dado pela regra: \\[2pt]
        \begin{equation}
            2^n \quad \text{linhas}
        \end{equation}
    \end{block}
\end{frame}
%
\subsection{Construção de uma tabela verdade}
%
\begin{frame}[t]
    \frametitle{Construção de uma tabela verdade}
    \framesubtitle{True table construction}
    %
    \begin{exampleblock}{Estudo de caso}
        Caso 1: $H(p,q) = \lnot (p~\land \lnot q)$
    \end{exampleblock}
    %
    \begin{table}[T]
        \caption{Composição de tabela verdade.}
        \label{tab:tabela-caso-1}
        \begin{tabular}{|c|c|c|l|l|}
        \hline
        \rowcolor[HTML]{EFEFEF} 
        \textbf{p} & \textbf{q} & \textbf{$\lnot q$} & \textbf{$p~\land \lnot q$} & \textbf{$\lnot (p~\land \lnot q)$} \\ \hline
        $V$ & $V$ &  &  &  \\ \hline
        $V$ & $F$ &  &  &  \\ \hline
        $F$ & $V$ &  &  &  \\ \hline
        $F$ & $F$ &  &  &  \\ \hline
        \end{tabular}
    \end{table}
\end{frame}
%
\begin{frame}[t]
    \frametitle{Construção de uma tabela verdade}
    \framesubtitle{True table construction}
    %
    \begin{exampleblock}{Estudo de caso}
        Caso 2: $G(p,q) = \lnot (p~\land \lnot q) \lor \lnot (q~\leftrightarrow p)$
    \end{exampleblock}
    %
    % \vspace{-4mm}
    % \hspace{-100mm}
    \small
    \begin{table}[ht]
        \caption{Composição de tabela verdade.}
        \label{tab:tabela-caso-2}
        \begin{tabular}{|c|c|c|l|l|l|l|}
        \hline
        \rowcolor[HTML]{EFEFEF} 
        \textbf{p} &
        \textbf{q} &
        \textbf{$p~\land \lnot q$} &
        \textbf{$q~\leftrightarrow p$} &
        \textbf{$\lnot (p~\land \lnot q)$} &
        \textbf{$\lnot (q~\leftrightarrow p)$} &
        \textbf{$\lnot (p~\land q) \lor \lnot (q~\leftrightarrow p)$} \\ \hline
        $V$ & $V$ &  &  &  &  &  \\ \hline
        $V$ & $F$ &  &  &  &  &  \\ \hline
        $F$ & $V$ &  &  &  &  &  \\ \hline
        $F$ & $F$ &  &  &  &  &  \\ \hline
        \end{tabular}
    \end{table}
\end{frame}
%

\end{document}