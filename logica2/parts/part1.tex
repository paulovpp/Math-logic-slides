%% Section 1 - Tabela verdade
\section{Tabela verdade}
%
\subsection{Definições iniciais}
%
\begin{frame}[c]
    \setbeamercolor{part title}{fg=white, bg=blue!65!black}
    \begin{beamercolorbox}[rounded=true,shadow=true,sep=12pt,center]{part title}
        \usebeamerfont{section title}\insertsection\par
    \end{beamercolorbox}
\end{frame}
%
\begin{frame}[t]
    \frametitle{Definições iniciais}
    \framesubtitle{Introductory definitions to the topic}
    %
    \begin{tcolorbox}[colback=blue!5,colframe=blue!60!black,adjusted title=Número de linhas]
        O número de linhas de uma tabela verdade de uma proposição composta depende do número $n$ de proposições simples que a integram sendo dado pela regra:
        \begin{equation}
            2^n \quad \text{linhas}
        \end{equation}
        %
        \tcblower
        %
        Para $n$ proposições simples do tipo $p_1,~p_2,\dots,p_n$, então a tabela verdade deve possuir um total de $n$ colunas para as proposições simples e $2^n$ linhas. Posto isso:
        \begin{itemize}
            \item Para a 1\textsuperscript{\d a} proposição simples $p_1$ atribui-se $2^n / 2^1 = 2^{n-1}$ valores $V$ seguidos de $F$ na mesma proporção.
        \end{itemize}
    \end{tcolorbox}
\end{frame}
%
\begin{frame}[t]
    \frametitle{Definições iniciais}
    \framesubtitle{Introductory definitions to the topic}
    %
    \begin{tcolorbox}[colback=blue!5,colframe=blue!60!black,adjusted title=Número de linhas]
        \begin{itemize}
            \item Para a 2\textsuperscript{\d a} proposição simples $p_2$ atribui-se $2^n / 2^2 = 2^{n-2}$ valores $V$ seguidos de $F$ na mesma proporção, repetindo-se até o final da tabela.
            \item Para a 3\textsuperscript{\d a} proposição simples $p_3$ atribui-se $2^n / 2^3 = 2^{n-3}$ valores $V$ seguidos de $F$ na mesma proporção, repetindo-se até o final da tabela.
            \item De modo genérico, para a k-ésima proposição simples $p_k (k\leq n)$ atribui-se \textbf{alternadamente} 
            \begin{equation}
                2^n / 2^k = 2^{n-k}
            \end{equation}
            \vspace{-2pt}
            valores $V$ seguidos de igual número de valores $F$, repetindo a sequência até o final das linhas da tabela verdade.
        \end{itemize}
    \end{tcolorbox}
\end{frame}
%
\subsection{Construção de uma tabela verdade}
%
{
\setbeamercolor{background canvas}{bg=yellow!15!white}
\begin{frame}[t]
    \frametitle{Construção de uma tabela verdade}
    \framesubtitle{True table construction}
    %
    \begin{exampleblock}{}
        \textbf{Caso 1:} $H(p,q) = \lnot (p~\land \lnot q)$
    \end{exampleblock}
    %
    \begin{table}[T]
        \caption{Tabela verdade para uma proposição composta $H(p,q)$.}
        \label{tab:tabela-caso-1}
        \begin{tabular}{|c|c|c|c|c|}
            % \tiny
            \hline
            \rowcolor[HTML]{EFEFEF} 
            \textbf{p} & \textbf{q} & $\lnot q$ & $p~\land \lnot q$ & $\lnot (p~\land \lnot q)$ \\ \hline
            $V$ & $V$ & $F$ & $F$ & $V$ \\ \hline
            $V$ & $F$ & $V$ & $V$ & $F$ \\ \hline
            $F$ & $V$ & $F$ & $F$ & $V$ \\ \hline
            $F$ & $F$ & $V$ & $F$ & $V$ \\ \hline
        \end{tabular}
    \end{table}
\end{frame}
}
%
\begin{frame}[t]
    \frametitle{Construção de uma tabela verdade}
    \framesubtitle{True table construction}
    %
    \begin{exampleblock}{}
        \textbf{Caso 2:} $G(p,q) = \lnot (p~\land \lnot q) \lor \lnot (q~\leftrightarrow p)$
    \end{exampleblock}
    %
    % \vspace{-4mm}
    % \hspace{-10mm}
    %
    \begin{table}[ht]
        \footnotesize
        \caption{Tabela verdade para uma proposição $G(p,q)$.}
        \label{tab:tabela-caso-2}
        \begin{tabular}{|c|c|c|c|c|c|c|}
            \hline
            \rowcolor[HTML]{EFEFEF} 
            \textbf{p} &
            \textbf{q} &
            \textbf{$p~\land \lnot q$} &
            \textbf{$q \leftrightarrow p$} &
            \textbf{$\lnot (p~\land \lnot q)$} &
            \textbf{$\lnot (q \leftrightarrow p)$} &
            \textbf{$\lnot (p~\land q) \lor \lnot (q~\leftrightarrow p)$} \\ \hline
            $V$ & $V$ & $F$ & $V$ & $V$ & $F$ & $V$ \\ \hline
            $V$ & $F$ & $V$ & $F$ & $F$ & $V$ & $V$ \\ \hline
            $F$ & $V$ & $F$ & $F$ & $V$ & $V$ & $V$ \\ \hline
            $F$ & $F$ & $F$ & $V$ & $V$ & $F$ & $V$ \\ \hline
        \end{tabular}
    \end{table}
    \begin{center}
        \textbf{\textcolor{red}{Proposição Tautológica.}}    
    \end{center}
\end{frame}
%
{
\setbeamercolor{background canvas}{bg=yellow!15!white}
\begin{frame}[t]
    \frametitle{Construção de uma tabela verdade}
    \framesubtitle{True table construction}
    %
    \begin{exampleblock}{}
        \textbf{Caso 3:} $P(p,q,r) = (p \rar (\n q \ou r)) \e \n (q \ou(p \lr \n r))$
    \end{exampleblock}
    %
    \begin{table}[]
        \caption{Tabela verdade para uma proposição $P(p,q,r)$.}
        \label{tab:tabela-caso-3}
        \renewcommand{\tabcolsep}{3pt}
        \begin{tabular}{cccc|c|c|c|c}
        \cline{5-5} \cline{7-7}
        \multicolumn{1}{l}{}                                     & \multicolumn{1}{l}{}                                    & \multicolumn{1}{l}{}                                    & \multicolumn{1}{l|}{} & \textbf{A}                         & \multicolumn{1}{l|}{}       & \textbf{B}                                      & \multicolumn{1}{l}{}                                     \\ \hline
        \rowcolor[HTML]{EFEFEF} 
        \multicolumn{1}{|c|}{\cellcolor[HTML]{EFEFEF}\textbf{p}} & \multicolumn{1}{c|}{\cellcolor[HTML]{EFEFEF}\textbf{q}} & \multicolumn{1}{c|}{\cellcolor[HTML]{EFEFEF}\textbf{r}} & $\lnot q \lor r$      & $(p \rightarrow (\lnot q \lor r))$ & $p \leftrightarrow \lnot r$ & \textbf{$(q \lor (p \leftrightarrow \lnot r))$} & \multicolumn{1}{c|}{\cellcolor[HTML]{EFEFEF}$A \land B$} \\ \hline
        \multicolumn{1}{|c|}{$V$}                                & \multicolumn{1}{c|}{$V$}                                & \multicolumn{1}{c|}{$V$}                                & $V$                   & $V$                                & $F$                         & $F$                                             & \multicolumn{1}{c|}{$F$}                                 \\ \hline
        \multicolumn{1}{|c|}{$V$}                                & \multicolumn{1}{c|}{$V$}                                & \multicolumn{1}{c|}{$F$}                                & $F$                   & $F$                                & $V$                         & $V$                                             & \multicolumn{1}{c|}{$F$}                                 \\ \hline
        \multicolumn{1}{|c|}{$V$}                                & \multicolumn{1}{c|}{$F$}                                & \multicolumn{1}{c|}{$V$}                                & $V$                   & $V$                                & $F$                         & $F$                                             & \multicolumn{1}{c|}{$F$}                                 \\ \hline
        \multicolumn{1}{|c|}{$V$}                                & \multicolumn{1}{c|}{$F$}                                & \multicolumn{1}{c|}{$F$}                                & $V$                   & $V$                                & $V$                         & $F$                                             & \multicolumn{1}{c|}{$F$}                                 \\ \hline
        \multicolumn{1}{|c|}{$F$}                                & \multicolumn{1}{c|}{$V$}                                & \multicolumn{1}{c|}{$V$}                                & $V$                   & $V$                                & $V$                         & $V$                                             & \multicolumn{1}{c|}{$V$}                                 \\ \hline
        \multicolumn{1}{|c|}{$F$}                                & \multicolumn{1}{c|}{$V$}                                & \multicolumn{1}{c|}{$F$}                                & $F$                   & $V$                                & $F$                         & $F$                                             & \multicolumn{1}{c|}{$F$}                                 \\ \hline
        \multicolumn{1}{|c|}{$F$}                                & \multicolumn{1}{c|}{$F$}                                & \multicolumn{1}{c|}{$V$}                                & $V$                   & $V$                                & $V$                         & $F$                                             & \multicolumn{1}{c|}{$F$}                                 \\ \hline
        \multicolumn{1}{|c|}{$F$}                                & \multicolumn{1}{c|}{$F$}                                & \multicolumn{1}{c|}{$F$}                                & $V$                   & $V$                                & $F$                         & $F$                                             & \multicolumn{1}{c|}{$F$}                                 \\ \hline
        \end{tabular}
    \end{table}
\end{frame}
}
% 