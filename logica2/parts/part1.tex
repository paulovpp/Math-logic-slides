%% Section 1 - Tabela verdade
\section{Tabela verdade}
%
\subsection{Definições iniciais}
%
\begin{frame}[c]
    \frametitle{Definições iniciais}
    \framesubtitle{Introductory definitions to the topic}
    \begin{block}{Número de linhas}
        O número de linhas de uma tabela verdade de uma proposição composta depende do número $n$ de proposições simples que a integram sendo dado pela regra: \\[2pt]
        \begin{equation}
            2^n \quad \text{linhas}
        \end{equation}
    \end{block}
    %
    \begin{tcolorbox}[colback=red!5,colframe=red!75!black,title=My title]
        My cool formalization
        \tcblower
        \begin{equation}
            \displaystyle\sum\limits_{i=1}^n i = \frac{n(n+1)}{2}    
        \end{equation}
        \textcolor{red}{Text in RED!!}
    \end{tcolorbox}
\end{frame}
%
\subsection{Construção de uma tabela verdade}
%
{
\setbeamercolor{background canvas}{bg=black!15!white}
\begin{frame}[t]
    \frametitle{Construção de uma tabela verdade}
    \framesubtitle{True table construction}
    %
    \begin{exampleblock}{Estudo de caso}
        Caso 1: $H(p,q) = \lnot (p~\land \lnot q)$
    \end{exampleblock}
    %
    \begin{table}[T]
        \caption{Composição de tabela verdade.}
        \label{tab:tabela-caso-1}
        \begin{tabular}{|c|c|c|l|l|}
            % \tiny
            \hline
            \rowcolor[HTML]{EFEFEF} 
            \textbf{p} & \textbf{q} & \textbf{$\lnot q$} & \textbf{$p~\land \lnot q$} & \textbf{$\lnot (p~\land \lnot q)$} \\ \hline
            $V$ & $V$ &  &  &  \\ \hline
            $V$ & $F$ &  &  &  \\ \hline
            $F$ & $V$ &  &  &  \\ \hline
            $F$ & $F$ &  &  &  \\ \hline
        \end{tabular}
    \end{table}
\end{frame}
}
%
\begin{frame}[t]
    \frametitle{Construção de uma tabela verdade}
    \framesubtitle{True table construction}
    %
    \begin{exampleblock}{Estudo de caso}
        Caso 2: $G(p,q) = \lnot (p~\land \lnot q) \lor \lnot (q~\leftrightarrow p)$
    \end{exampleblock}
    %
    % \vspace{-4mm}
    % \hspace{-10mm}
    %
    \begin{table}[ht]
        \footnotesize
        \caption{Composição de tabela verdade.}
        \label{tab:tabela-caso-2}
        \begin{tabular}{|c|c|c|l|l|l|l|}
            \hline
            \rowcolor[HTML]{EFEFEF} 
            \textbf{p} &
            \textbf{q} &
            \textbf{$p~\land \lnot q$} &
            \textbf{$q \leftrightarrow p$} &
            \textbf{$\lnot (p~\land \lnot q)$} &
            \textbf{$\lnot (q \leftrightarrow p)$} &
            \textbf{$\lnot (p~\land q) \lor \lnot (q~\leftrightarrow p)$} \\ \hline
            $V$ & $V$ &  &  &  &  &  \\ \hline
            $V$ & $F$ &  &  &  &  &  \\ \hline
            $F$ & $V$ &  &  &  &  &  \\ \hline
            $F$ & $F$ &  &  &  &  &  \\ \hline
        \end{tabular}
    \end{table}
\end{frame}
%