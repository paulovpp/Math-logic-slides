% Section 4 - Propriedades semânticas 2
\section{Propriedades semânticas 2}
%
\subsection{Implicação lógica}
%
\begin{frame}[c]
    \setbeamercolor{part title}{fg=white, bg=blue!65!black}
    \begin{beamercolorbox}[rounded=true,shadow=true,sep=12pt,center]{part title}
        \usebeamerfont{section title}\insertsection\par
    \end{beamercolorbox}
\end{frame}
%
\begin{frame}[t]
    \frametitle{Implicação lógica}
    \framesubtitle{Brief concepts over logical implication}
    %
    \begin{tcolorbox}[colback=blue!5,colframe=blue!60!black,adjusted title=Definição]
        \indent Dada uma proposição $P(p,q,r,\dots)$; diz-se que ela implica logicamente em $Q(p,q,r,\dots)$ se para todo valor lógico verdadeiro (V) de $P(p,q,r,\dots)$ também for verdadeiro em $Q(p,q,r,\dots)$.
        %
        \tcblower
        %
        \indent Indica-se que a proposição $P(p,q,r,\dots)$ \textbf{implica} a proposição $Q(p,q,r,\dots)$ com a notação: $$P(p,q,r,\dots) \Rightarrow Q(p,q,r,\dots)$$
        \indent Em particular, toda proposição implica uma tautologia e somente uma contradição implica uma contradição.
    \end{tcolorbox}
    %    
    % \begin{tcolorbox}[colback=red!5!white,colframe=red!75!black,title=Descritivo]
    %     \textbf{Sintaxe:} responsável pelo conjunto de símbolos (ALFABETO), conectivos e figuras utilizados pela lógica.
    %     %
    % \end{tcolorbox}
\end{frame}
%
\subsection{Propriedades e exemplos}
%
\begin{frame}[t]
    \frametitle{Propriedades da implicação lógica}
    \framesubtitle{Logical implication properties}
    %
    \setbeamercolor{block title}{use=structure,fg=white,bg=black!75!black}
    \begin{block}{\centering Reflexiva}
        \begin{equation}
            P(p,q,r,\dots) \rr P(p,q,r,\dots)
        \end{equation}
    \end{block}
    %
    \begin{block}{\centering Transitiva}
        \vspace{-4mm}
        \begin{align}
            &\text{Se\quad} P(p,q,r,\dots) \rr Q(p,q,r,\dots) \text{ e}\\ \nonumber
            &P(p,q,r,\dots) \rr R(p,q,r,\dots) \text{, então} \\
            &P(p,q,r,\dots) \rr R(p,q,r,\dots)
        \end{align}
    \end{block}
\end{frame}
%
\begin{frame}[t]
    \frametitle{Propriedades da implicação lógica}
    \framesubtitle{Logical implication properties}
    %
    \begin{exampleblock}{1. Regras de inferência}
        As tabelas verdade das proposições:
        $$ P_1(p,q) = p \e q, \qquad  P_2(p,q) = p \ou q, \qquad P_3(p,q) = p \lr q$$
        são:
    \end{exampleblock}
    %
    \vspace{-4mm}
    %
    \begin{columns}
        \begin{column}{0.5\textwidth}
            \small
            \begin{table}[]
                \caption{Tabela verdade das proposições $P_1$, $P_2$ e $P_3$.}
                \label{tab:exemple-31}
                \vspace{-2mm}
                \begin{tabular}{cc||c|c|c|}
                \cline{3-5}
                \multicolumn{1}{l}{}                                     & \multicolumn{1}{l|}{} & $P_1$    & $P_2$     & $P_3$     \\ \hline
                \rowcolor[HTML]{EFEFEF} 
                \multicolumn{1}{|c|}{\cellcolor[HTML]{EFEFEF}\textbf{p}} & \textbf{q}            & $p \e q$ & $p \ou q$ & $p \lr q$ \\ \hline
                \multicolumn{1}{|c|}{$V$}                                & $V$                   & $V$      & $V$       & $V$       \\ \hline
                \multicolumn{1}{|c|}{$V$}                                & $F$                   & $F$      & $V$       & $F$       \\ \hline
                \multicolumn{1}{|c|}{$F$}                                & $V$                   & $F$      & $V$       & $F$       \\ \hline
                \multicolumn{1}{|c|}{$F$}                                & $F$                   & $F$      & $F$       & $V$       \\ \hline
                \end{tabular}
            \end{table}
        \end{column}
        %
        \hspace{-6mm}
        %
        \begin{column}{0.4\textwidth}
            \setbeamercolor{enumerate item}{fg=red!80!black}
            \setbeamertemplate{enumerate items}[default]
            \begin{enumerate}[(i)]
                \item $p \e q \rr p \ou q$
                \item $p \e q \rr p \lr q$
                \item $p \rr p \ou q ~e~ q \rr p \ou q$
                \item $p \e q \rr p ~e~ p \e q \rr q$
            \end{enumerate}
        \end{column}
    \end{columns}
\end{frame}
%
\begin{frame}[t]
    \frametitle{Propriedades da implicação lógica}
    \framesubtitle{Logical implication properties}
    %
    \begin{exampleblock}{2. Regra do Silogismo disjuntivo}
        As tabelas verdade das proposições:
        $$ P_1(p,q) = p \e q, \qquad  P_2(p,q) = p \ou q, \qquad P_3(p,q) = p \lr q$$
        são:
    \end{exampleblock}
    %
    \vspace{-4mm}
    %
    \begin{columns}
        \begin{column}{0.5\textwidth}
            \small
            \begin{table}[]
                \caption{Tabela verdade das proposições $P_1$, $P_2$ e $P_3$.}
                \label{tab:exemple-31}
                \vspace{-2mm}
                \begin{tabular}{cc||c|c|c|}
                \cline{3-5}
                \multicolumn{1}{l}{}                                     & \multicolumn{1}{l|}{} & $P_1$    & $P_2$     & $P_3$     \\ \hline
                \rowcolor[HTML]{EFEFEF} 
                \multicolumn{1}{|c|}{\cellcolor[HTML]{EFEFEF}\textbf{p}} & \textbf{q}            & $p \e q$ & $p \ou q$ & $p \lr q$ \\ \hline
                \multicolumn{1}{|c|}{$V$}                                & $V$                   & $V$      & $V$       & $V$       \\ \hline
                \multicolumn{1}{|c|}{$V$}                                & $F$                   & $F$      & $V$       & $F$       \\ \hline
                \multicolumn{1}{|c|}{$F$}                                & $V$                   & $F$      & $V$       & $F$       \\ \hline
                \multicolumn{1}{|c|}{$F$}                                & $F$                   & $F$      & $F$       & $V$       \\ \hline
                \end{tabular}
            \end{table}
        \end{column}
        %
        \hspace{-6mm}
        %
        \begin{column}{0.4\textwidth}
            \setbeamercolor{enumerate item}{fg=red!80!black}
            \setbeamertemplate{enumerate items}[default]
            \begin{enumerate}[(i)]
                \item $p \e q \rr p \ou q$
                \item $p \e q \rr p \lr q$
                \item $p \rr p \ou q ~e~ q \rr p \ou q$
                \item $p \e q \rr p ~e~ p \e q \rr q$
            \end{enumerate}
        \end{column}
    \end{columns}
\end{frame}