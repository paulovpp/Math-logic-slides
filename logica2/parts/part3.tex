% Section 4 - Propriedades semânticas 2
\section{Propriedades semânticas 2}
%
\subsection{Implicação lógica}
%
\begin{frame}[c]
    \setbeamercolor{part title}{fg=white, bg=blue!65!black}
    \begin{beamercolorbox}[rounded=true,shadow=true,sep=12pt,center]{part title}
        \usebeamerfont{section title}\insertsection\par
    \end{beamercolorbox}
\end{frame}
%
\begin{frame}[t]
    \frametitle{Algumas definições}
    \framesubtitle{Some definitions to the topic}
    %
    \begin{tcolorbox}[colback=blue!5,colframe=blue!60!black,adjusted title=Mundo lógico]
        O mundo da lógica conforme conhecemos pode ser dividido em duas partes distintas a seguir:
        \begin{itemize}
            \item Sintaxe - mundo sintático
            \item Semântica - mundo semântico
        \end{itemize}
    \end{tcolorbox}
    %    
    \begin{tcolorbox}[colback=red!5!white,colframe=red!75!black,title=Descritivo]
        \textbf{Sintaxe:} responsável pelo conjunto de símbolos (ALFABETO), conectivos e figuras utilizados pela lógica.
        %
        \tcblower
        %
        \textbf{Semântica:} responsável pelas operações e regras de forma a utilizar da melhor forma possível o conjunto de símbolos.
    \end{tcolorbox}
\end{frame}
%