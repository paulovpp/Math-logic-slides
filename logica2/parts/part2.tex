%% Section 2 - Mundo da lógica proposicional
\section{Lógica proposicional - sintaxe e semântica}
%
\subsection{Mundo da lógica proposicional}
%
\begin{frame}[c]
    \setbeamercolor{part title}{fg=white, bg=blue!65!black}
    \begin{beamercolorbox}[rounded=true,shadow=true,sep=12pt,center]{part title}
        \usebeamerfont{section title}\insertsection\par
    \end{beamercolorbox}
\end{frame}
%
\begin{frame}[t]
    \frametitle{Algumas definições}
    \framesubtitle{Some definitions to the topic}
    %
    \begin{tcolorbox}[colback=blue!5,colframe=blue!60!black,adjusted title=Mundo lógico]
        O mundo da lógica conforme conhecemos pode ser dividido em duas partes distintas a seguir:
        \begin{itemize}
            \item Sintaxe - mundo sintático
            \item Semântica - mundo semântico
        \end{itemize}
    \end{tcolorbox}
    %    
    \begin{tcolorbox}[colback=red!5!white,colframe=red!75!black,title=Descritivo]
        \textbf{Sintaxe:} responsável pelo conjunto de símbolos (ALFABETO), conectivos e figuras utilizados pela lógica.
        %
        \tcblower
        %
        \textbf{Semântica:} responsável pelas operações e regras de forma a utilizar da melhor forma possível o conjunto de símbolos.
    \end{tcolorbox}
\end{frame}
%
\begin{frame}[t]
    \frametitle{Algumas definições}
    \framesubtitle{Some definitions to the topic}
    %
    \begin{block}{Na prática}
        \begin{itemize}
            \item O computador é um aparelho extremamente sintático - opera com a representação de símbolos em linguagem de máquina, \textbf{baixo nível} e com a possibilidade de conversão dos mesmos para um nível inteligível aos seres humanos, conhecido como \textbf{alto nível}.
            \item Para que o computador possa desempenhar suas funções, um conjunto de regras (ALGORITMO) precisa ser definido, enviado e traduzido para sua interpretação e execução.
        \end{itemize}
    \end{block}
    %
    \setbeamercolor{block title}{use=structure,fg=black,bg=yellow!75!black}
    \begin{block}{Regras ou significados}
        Caso a definição ou o significado de um conjunto de símbolos não seja bem definido, falhas de semântica podem ocorrer. Isso não fará com que não haja processamento. Porém, o resultado pode não ser o esperado.
    \end{block}
\end{frame}
%
\begin{frame}[t]
    \frametitle{Algumas definições}
    \framesubtitle{Some definitions to the topic}
    %
    \begin{block}{Exemplo de falha semântica}
        Observe o seguinte conjunto de caracteres - símbolos sintáticos:
        \begin{center}
            \textbf{R  E  D  E}
        \end{center}
    \end{block}
        %
    \begin{exampleblock}{}
        \indent Caso o uso do seguinte conjunto de símbolos seja utilizado sem a prévia e correta definição de sua usabilidade, poderá haver uma falha de execução e resultados discrepantes. Observa-se pelo menos três possíveis usos da palavra acima:
        \begin{itemize}
            \item objeto usado para dormir.
            \item objeto usado para pescar.
            \item descrição de um conjunto de computadores.
        \end{itemize}
    \end{exampleblock}
\end{frame}
%
\subsection{Valor lógico de uma proposição composta}
%
\begin{frame}[t]
    \frametitle{Valor lógico de uma proposição composta}
    \framesubtitle{Logical values (interpretations) for compound propositions}
    %
    \begin{tcolorbox}[colback=red!5!white,colframe=red!75!black,title=Definição]
        \indent Dado uma proposição composta $H(p,q, r, \dots)$ pode-se determinar seu valor lógico, V ou F, quando são conhecidos os valores lógicos de suas proposições simples respectivamente.
        %
        \tcblower
        %
        \textbf{Exemplo 1:}
        \begin{tcolorbox}
            Assumindo $P(p,q) = (p \rar q) \rar (p \rar p\e q)$, calcule: \\
            $V(P)$ quando $V(p)=V(q)=F$: \\ [4pt]
            $V(P(F,F))=V(P)=(F \rar F) \rar (F \rar F\e F)$ \\
            $V(P)=V \rar (F \rar F)$ \\
            $V(P) = V$
        \end{tcolorbox}
    \end{tcolorbox}
\end{frame}
%
\begin{frame}[t]
    \frametitle{Valor lógico de uma proposição composta}
    \framesubtitle{Logical values (interpretations) for compound propositions}
    %
    \begin{tcolorbox}[colback=red!5!white,colframe=red!75!black,title=Definição]
        \textbf{Exemplo 2:}
        \begin{tcolorbox}
            Dado:
            $$P(p,q,r) = (q \lr (r \rar \n p)) \ou ((\n q \rar p) \lr r)$$
            Calcule $V(P)$ quando $V(p)=V$ e $V(q)=V(r)=F$. \\ [4pt]
            $V(P(VFF)) = (F \lr (F \rar \n V)) \ou ((\n F \rar V) \lr F)$ \\
            $V(P) = (F \lr (F \rar F)) \ou ((V \rar V) \lr F)$ \\
            $V(P) = (F \lr V) \ou (V \lr F)$ \\
            $V(P) = (F) \ou (F)$ \\
            $V(P) = F$
        \end{tcolorbox}
    \end{tcolorbox}
\end{frame}
%
\subsection{Uso de parêntesis}
%
\begin{frame}[t]
    \frametitle{Uso de parêntesis}
    \framesubtitle{Parentheses use}
    %
    \setbeamercolor{block title}{use=structure,fg=white,bg=black!75!black}
    \setbeamercolor{enumerate item}{fg=red!80!black}
    \setbeamertemplate{enumerate items}[default]
    \begin{block}{Definição}
        \indent É óbvia a necessidade do uso dos parêntesis na simbolização das proposições e fórmulas. Muito utilizados para evitar qualquer tipo de ambiguidade. Assim, p. ex., a expressão $p \e q \ou r$ dá lugar, colocando parêntesis, às duas proposições a seguir:
        
        \begin{center}
            (i) $(p \e q) \ou r$ \quad e \quad (ii) $p \e (q \ou r)$
        \end{center}

        \indent Percebe-se aqui que ambas não possuem o mesmo significado pois para ambos os casos os conectivos principais são diferentes. \\ [4pt]
        \indent A supressão de parêntesis nas proposições se faz mediante algumas \textbf{convenções}, mostradas a seguir:
        \begin{enumerate}[\bf I. ]
            \item "A ordem de precedência" para os conectivos.
            \vspace{-2mm}
            $$(1) \sim  \quad (2) \e \ou  \quad (3) \rar \quad (4) \lr$$
        \end{enumerate} 
    \end{block}
\end{frame}
%
\begin{frame}[t]
    \frametitle{Uso de parêntesis}
    \framesubtitle{Parentheses use}
    %
    \setbeamercolor{block title}{use=structure,fg=white,bg=black!75!black}
    \setbeamercolor{enumerate item}{fg=red!80!black}
    \setbeamertemplate{enumerate items}[default]
    \begin{block}{Definição}
        \indent O conectivo mais fraco é, portanto o $\n$. E o mais forte $\lr$. Para o caso abaixo:
        $$ p \rar q \lr s \e r $$
        \indent Temos então uma \textcolor{red}{BICONDICIONAL} e nunca uma condicional. Para convertê-la em uma condicional, o uso dos parêntesis se faz necessário:
        $$ p \rar (q \lr s \e r) $$
        \begin{enumerate}[\bf II. ]
            \item Quando um mesmo conectivo aparece sucessivamente repetido, suprimem-se os parêntesis, fazendo-se a \textbf{associação} a partir da esquerda. Observe as seguintes proposições:
        \end{enumerate} 
    \end{block}
\end{frame}
%
%
\begin{frame}[t]
    \frametitle{Uso de parêntesis}
    \framesubtitle{Parentheses use}
    %
    \setbeamercolor{enumerate item}{fg=red!80!black}
    \setbeamertemplate{enumerate items}[default]
    \begin{exampleblock}{Exemplos}
        \begin{enumerate}[\bf 1.]
            \item $((\n (\n(p \e q))) \ou (\n p))$ \qquad $\longrightarrow$ \qquad $\n \n (p \e q) \ou \n p$
            \vspace{2mm}
            \item $(((p \e (\n q)) \ou r) \e (\n p))$ \qquad $\longrightarrow$ \qquad $(p \ou \n q) \e r \e \n p$
            \vspace{2mm}
            \item $((p \ou (\n q)) \e (r \e (\n p)))$ \qquad $\longrightarrow$ \qquad $(p \e \n q) \e (r \e \n p)$
            \vspace{2mm}
            \item $((\n p) \rar (q \rar (\n (p \ou r))))$ \qquad $\longrightarrow$ \qquad $\n p \rar (q \rar \n (p \ou r))$
        \end{enumerate}
    \end{exampleblock}
    %
    \begin{alertblock}{Outros símbolos para os conectivos}
        Também muito utilizado em linguagens de programação:
        \vspace{-2mm}
        \begin{align*}
            &' \sim '                      &     & \text{para negação.}  \\
            &' \centerdot '~\text{e}~'\&'  &     & \text{para conjunção.} \\
            &' \supset '                   &     & \text{(ferradura) para condicional.} 
        \end{align*}
    \end{alertblock}
\end{frame}

\section{Propriedades semânticas}
%
\subsection{Tautologia}
%
\begin{frame}[c]
    \setbeamercolor{part title}{fg=white, bg=blue!65!black}
    \begin{beamercolorbox}[rounded=true,shadow=true,sep=12pt,center]{part title}
        \usebeamerfont{section title}\insertsection\par
    \end{beamercolorbox}
\end{frame}
%
\begin{frame}[t]
    \frametitle{Tautologia}
    \framesubtitle{Tautology definitions}
    %
    \begin{tcolorbox}[colback=red!5!white,colframe=red!75!black,title=Definição]
        Dada uma fórmula $H(p,q,r,\dots)$ ela será uma tautologia quando para qualquer valor lógico de suas proposições simples, seu valor lógico será \textbf{sempre} verdadeiro (V). Ou seja:
        $$ \forall V(p,q,r,\dots) \to V(H)=V$$
        %
        \tcblower
        %
        \textbf{Exemplos:}
        \begin{align*}
            &H_1(p)   = p \ou \n p                  &        & H_2(p) = \n (p \e \n p) \\
            &H_3(p,q) = p \ou \n (p \e q)           &        & H_4(p,q) = p \e q \rar (p \lr p) \\
            &H_5(p,q) = p \ou (q \e \n q) \lr p     &        & H_6(p,q,r) = p \e r \rar \n q \ou r
        \end{align*}
    \end{tcolorbox}
\end{frame}
%
\begin{frame}[t]
    \frametitle{Tautologia - Exemplo}
    \framesubtitle{Tautology example}
    %
    \begin{exampleblock}{Exemplo}
        \centering
        $H_6(p,q,r) = p \e r \rar \n q \ou r $
        %
        \vspace{-2mm}
        %
        \begin{table}[t]
            \caption{Tabela verdade para uma proposição $P(p,q,r)$ tautológica.}
            \label{tab:exemple-tautology}
            \begin{tabular}{|c|c|c|c|c|c|c|}
            \hline
            \textbf{p} & \textbf{q} & \textbf{r} & $\n q$ & $p \e r$ & $\n q \ou r$ & $p \e r \rar \n q \ou r$ \\ \hline
            $V$        & $V$        & $V$        & $F$    & $V$      & $V$          & $V$                    \\ \hline
            $V$        & $V$        & $F$        & $F$    & $F$      & $F$          & $V$                    \\ \hline
            $V$        & $F$        & $V$        & $V$    & $V$      & $V$          & $V$                    \\ \hline
            $V$        & $F$        & $F$        & $V$    & $F$      & $V$          & $V$                    \\ \hline
            $F$        & $V$        & $V$        & $F$    & $F$      & $V$          & $V$                    \\ \hline
            $F$        & $V$        & $F$        & $F$    & $F$      & $F$          & $V$                    \\ \hline
            $F$        & $F$        & $V$        & $V$    & $F$      & $V$          & $V$                    \\ \hline
            $F$        & $F$        & $F$        & $V$    & $F$      & $V$          & $V$                    \\ \hline
            \end{tabular}
        \end{table}
    \end{exampleblock}
\end{frame}
%
\subsection{Princípio da substituição para as tautologias}
%
\begin{frame}[t]
    \frametitle{Princípio da substituição para as tautologias}
    \framesubtitle{Tautological replacement principle}
    %
    \begin{tcolorbox}[colback=blue!5,colframe=blue!60!black,adjusted title=Definição]
        \indent Seja $H(p,q,r,\dots)$ uma proposição tautológica qualquer e sejam $P(p,q,r,\dots)$, $Q(p,q,r,\dots)$, $R(p,q,r,\dots)$, ... proposições quaisquer formadas a parir do mesmo conjunto de proposições simples.
        
        \tcblower
        
        Realizando a substituição das proposições compostas $P, Q, R, \dots$ em $H$, a nova proposição $$ H(P,Q,R,\dots)$$ também será uma tautologia quaisquer que sejam as proposições $P, Q, R, \dots$ .
    \end{tcolorbox}
\end{frame}
%
\subsection{Contradição}
%
\begin{frame}[t]
    \frametitle{Contradição}
    \framesubtitle{Contradiction definitions}
    %
    \begin{tcolorbox}[colback=red!5!white,colframe=red!75!black,title=Definição]
        Dada uma fórmula $H(p,q,r,\dots)$ ela será uma contradição quando para qualquer valor lógico de suas proposições simples, seu valor lógico será \textbf{sempre} falso (F). Ou seja:
        $$ \forall V(p,q,r,\dots) \to V(H)=F$$
        As contradições também são conhecidas como proposições contraválidas ou logicamente falsas.
        %
        \tcblower
        %
        \textbf{Exemplos:}
        \begin{align*}
            &H_7(p) = p \e \n p                     &        & H_8(p) = p \lr \n p \\
            &H_9(p,q) = (p \e q) \e \n (p \ou q)    &        & H_{10}(p,q) = \n p \e (p \e \n q)
        \end{align*}
    \end{tcolorbox}
\end{frame}
%
\subsection{Contingência}
%
\begin{frame}[t]
    \frametitle{Contingência}
    \framesubtitle{Contigency definitions}
    %
    \begin{alertblock}{Definição}
        \indent Proposições que não são nem tautológicas ou contraválidas chamam-se necessariamente de contingentes. As mesmas possuem em sua última coluna de sua tabela-verdade os valores lógicos V e F \textbf{pelo menos uma vez}.\\
        \hfill \break
        \indent Elas também são conhecidas como \textbf{proposições indeterminadas}. \\
        {\color{red} \rule{\linewidth}{0.5pt}}
        \textbf{Exemplos:}
        \begin{align*}
            &H_{11}(p) = p \rar \n p                     &        & H_{12}(p,q) = p \ou q \rar p \e p \\
            &H_{13}(p,q) = p \ou q \rar p                &        & H_{14}(p,q) = p \rar (p \rar q \e \n q)
        \end{align*}
    \end{alertblock}
\end{frame}