%% Section 1 - Introductory concepts
\section{Introdução}
%
% \begin{frame}[c]
%     \begin{block}{\LARGE{Definições iniciais}}
%     \end{block}
% \end{frame}
\subsection{Objetivos}
%
\begin{frame}[c]
    \frametitle{Objetivos do curso}
    \framesubtitle{List all course objectives}
    \begin{alertblock}{Estudo da lógica proposicional}
        \begin{itemize}
            \item Representar e especificar os conceitos de sintaxe e semântica associados a qualquer lógica utilizada ou linguagem.
            \item Estudar os métodos que produzem ou verifiquem as fórmulas ou argumentos utilizados.
            \item Definir sistemas de dedução formal onde são consideradas as noções de prova e consequência lógica.
            \item Correlacionar diagramas de Venn com a prática.
            \item Conhecer a álgebra de Boole.
        \end{itemize}
    \end{alertblock}
\end{frame}
%
\subsection{Definições iniciais}
%
\begin{frame}[t]
    \frametitle{Definições iniciais}
    \framesubtitle{Introductory definitions to the course}
    \textbf{Proposição}\\
    \quad $\star$ \textcolor{blue}{É qualquer conjunto de palavras ou símbolos que expressam um pensamento completo.}\\
    \quad $\star$ As proposições transmitem fatos ou exprimem juízos que formamos a respeito de determinado acontecimento.\\ \pause
    \textbf{Exemplos}
    \begin{itemize}
        \item A lua é um satélite da Terra.\pause
        \item O valor arredondado de $\pi$ vale $3,14$.\pause
        \item Recife é a capital da Paraíba\pause
        \item $\cos(90^o)~=~0$.\pause
    \end{itemize}
    \textbf{Alfabeto}\\
    \quad $\star$ É o conjunto de símbolos usado em qualquer linguagem. A seguir a tabela de símbolos usados na disciplina é apresentado:
\end{frame}
%