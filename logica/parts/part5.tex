% Section 5 - Logical operations with conditions

\section{Operações lógicas condicionais}
%
\subsection{Condicional}
%
\begin{frame}[t]
    \frametitle{Operador condicional $(\rightarrow)$ }
    \framesubtitle{Conditional operator}
    %
    \vspace{-2mm}
    \begin{block}{Definição}
        Duas proposições formam uma condicional quando for possível colocá-las na seguinte forma: \\[2pt]
        \begin{center}
            \textit{Se (proposição 1), então (proposição 2)}
        \end{center}
        \begin{itemize}
            \item a proposição 1 é chamada de \textbf{antecedente}, e a proposição 2 de \textbf{consequente};
            \item o símbolo utilizado para ligar as duas proposições de uma condicional é o $(\rightarrow)$;
            \item sua representação é da forma $p \rightarrow q$ e lê-se $p$ então $q$.
            \item a equação acima também pode ser compreendida da forma:
            \begin{itemize}
                \item[\ding{114}] $p$ é condição suficiente para $q$
                \item[\ding{114}] $q$ é condição necessária para $p$
            \end{itemize}
        \end{itemize}
    \end{block}
\end{frame}
%
\begin{frame}[t]
    \frametitle{Operador condicional $(\rightarrow)$ }
    \framesubtitle{Conditional operator}
    %
    \begin{block}{Continuação \dots}
        \begin{itemize}
            \item as proposições condicionais podem ter sentidos diferentes em sua composição, veja os exemplos a seguir.
            \begin{itemize}
                \item[] $p:$ Alberto é poliglota.
                \item[] $q:$ Alberto(ele) fala várias línguas.
            \end{itemize}
            \item Simbolicamente temos $p \rightarrow q$ e lê-se: "Se Alberto é poliglota, ele fala várias línguas.
        \end{itemize}
    \end{block}
    %
    \vspace{-5mm}
    %
    \begin{table}[ht]
        \caption{Tabela verdade $(\rightarrow)$.}
        \label{tab:tabela-condicao}
        \begin{tabular}{|c|c|c|}
        \hline
        \rowcolor[HTML]{EFEFEF} 
        \textbf{p} & \textbf{q} & \textbf{p $\rightarrow$ q} \\ \hline
        $V$        & $V$        & $V$                        \\ \hline
        $V$        & $F$        & $F$                        \\ \hline
        $F$        & $V$        & $V$                        \\ \hline
        $F$        & $F$        & $V$                        \\ \hline
        \end{tabular}
    \end{table}
\end{frame}
%
\begin{frame}[t]
    \frametitle{Operador condicional $(\rightarrow)$ }
    \framesubtitle{Examples}
    %
    \small
    \begin{exampleblock}{Exemplos}
        \begin{itemize}
            \item[] $p:$ Santos Dumont é cearense.
            \item[] $q:$ Fevereiro tem 31 dias.
        \end{itemize}
        $p \rightarrow q:$ Se Santos Dumont é cearense, então Fevereiro tem 31 dias. $(V)$ \\[2pt]
        $V(p \rightarrow q) = V(p) \rightarrow V(q) = F \rightarrow F = V$
    \end{exampleblock}
    %
    \vspace{-2mm}
    %
    \begin{exampleblock}{}
        \begin{itemize}
            \item[] $r:$ Maio tem 31 dias.
            \item[] $s:$ A terra é plana. 
        \end{itemize}
        $r \rightarrow s:$ Se Maio tem 31 dias, então a terra não é plana. $(F)$ \\[2pt]
        $V(r \rightarrow r) = V(r) \rightarrow V(s) = V \rightarrow F = F$
    \end{exampleblock}
    %
    \vspace{-2mm}
    %
    \begin{alertblock}{}
        \textbf{NOTA:} uma função condicional $p \rightarrow q$ \textbf{não afirma} que o consequente $q$ se deduz ou é \textbf{consequência} do antecedente $p$.
    \end{alertblock}
\end{frame}
%
\subsection{Bicondicional}
%
\begin{frame}[t]
    \frametitle{Operador bicondicional $(\leftrightarrow)$ }
    \framesubtitle{Biconditional operator}
    %
    \begin{block}{Definição}
        Duas proposições formam uma bicondicional quando for possível colocá-las na seguinte forma: \\[2pt]
        \begin{center}
            \textit{(proposição 1) se, e somente se, (proposição 2)}
        \end{center}
        \begin{itemize}
            \item a proposição 1 é chamada de \textbf{antecedente}, e a proposição 2 de \textbf{consequente};
            \item o símbolo utilizado para ligar as duas proposições de uma bicondicional é o $(\leftrightarrow)$;
            \item sua representação é da forma $p \leftrightarrow q$ e lê-se $p$ se, e somente se, $q$.
            \item a equação acima também pode ser compreendida da forma:
            \begin{itemize}
                \item[\ding{114}] Se $p$, então $q$. $(p \rightarrow q)$
                \item[\ding{114}] Se $q$, então $p$. $(q \rightarrow p)$
            \end{itemize}
        \end{itemize}
    \end{block}
\end{frame}
%
\begin{frame}[t]
    \small
    \frametitle{Operador bicondicional $(\leftrightarrow)$ }
    \framesubtitle{Biconditional operator}
    %
    \setbeamercolor{enumerate item}{fg=red!80!black}
    \setbeamertemplate{enumerate items}[default]
    \vspace{-2mm}
    \begin{block}{Continuação\dots}
        Alternativamente, a \textbf{bicondicional de duas proposições} $p$ e $q$ também pode ser lida de uma das seguintes maneiras abaixo:
        \begin{enumerate}[(i)]
            \item $p$ é condição necessária e suficiente para $q$
            \item $q$ é condição necessária e suficiente para $p$
        \end{enumerate}
        De acordo com a Tabela \ref{tab:tabela-bicondicao} abaixo, o \textbf{valor lógico} da bicondicional será VERDADEIRO somente quando também o são as condicionais $p \rightarrow q$ e $q \rightarrow p$.
    \end{block}
    %
    \vspace{-4mm}
    %
    \begin{table}[ht]
        \caption{Tabela verdade $(\leftrightarrow)$.}
        \label{tab:tabela-bicondicao}
        \begin{tabular}{|c|c|c|}
        \hline
        \rowcolor[HTML]{EFEFEF} 
        \textbf{p} & \textbf{q} & \textbf{p $\leftrightarrow$ q} \\ \hline
        $V$        & $V$        & $V$                            \\ \hline
        $V$        & $F$        & $F$                            \\ \hline
        $F$        & $V$        & $F$                            \\ \hline
        $F$        & $F$        & $V$                            \\ \hline
        \end{tabular}
    \end{table}
\end{frame}
%
\subsection{Resumo}
%
\begin{frame}[t]
    \frametitle{Resumo dos operadores}
    \framesubtitle{Sections summary}

    \setbeamercolor{block title}{use=structure,fg=white,bg=purple!75!black}
    \begin{block}{Operações lógicas}
        \begin{table}[]
    \caption{Operações lógicas com proposições.}
    \label{tab:resumo}
    \begin{tabular}{|c|c|c|c|c|c|c|c|}
    \hline
    \rowcolor[HTML]{656565} 
    {\color[HTML]{FFFFFF} \textbf{$p$}} & {\color[HTML]{FFFFFF} \textbf{$q$}} & {\color[HTML]{FFFFFF} \textbf{$\lnot p$}} & {\color[HTML]{FFFFFF} \textbf{$\land$}} & {\color[HTML]{FFFFFF} \textbf{$\lor$}} & {\color[HTML]{FFFFFF} \textbf{$\veebar$}} & {\color[HTML]{FFFFFF} \textbf{$\rightarrow$}} & {\color[HTML]{FFFFFF} \textbf{$\leftrightarrow$}} \\ \hline
    $V$                               & $V$                               & $F$                                       & $V$                                     & $V$                                    & $F$                                       & $V$                                           & $V$                                               \\ \hline
    $V$                               & $F$                               & $F$                                       & $F$                                     & $V$                                    & $V$                                       & $F$                                           & $F$                                               \\ \hline
    $F$                               & $V$                               & $V$                                       & $F$                                     & $V$                                    & $V$                                       & $V$                                           & $F$                                               \\ \hline
    $F$                               & $F$                               & $V$                                       & $F$                                     & $F$                                    & $F$                                       & $V$                                           & $V$                                               \\ \hline
    \end{tabular}
    \end{table}
    \end{block}

\end{frame}