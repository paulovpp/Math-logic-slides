\documentclass[10pt]{beamer}
\usepackage[utf8]{inputenc}
\usepackage[T1]{fontenc}
\usepackage[brazilian]{babel}
\usepackage{xcolor}

\usetheme{Warsaw}
\usecolortheme{default}
\usefonttheme{professionalfonts}
\graphicspath{{./imgs/}}
\setbeamertemplate{caption}[numbered]
\setbeamercovered{transparent}
\geometry{left=1cm, right=1cm}

\title{Lógica Matemática} 
\author[Paulo Pinheiro]
{Dr. Paulo Vinicius Pereira Pinheiro\inst{1}}
\institute[UNIFAP]
{
    \inst{1}
    Centro Universitário Paraíso do Ceará\\
    UNIFAP
}
%
\date{\small{Acesse estes slides em:\\ \url{https://github.com/paulovpp/slides}}\newline \\Última atualização:\\ \today}
\logo{\includegraphics[height=0.8cm]{imgs/UNIFAP.png}}
%
%%%%%%%%%%%%%%%%%%%%%%%%%%%%%%%%%%%%%%%%%%%%%%%%%%%%%%%%%%%%%%%%%%%%%%%%%%%%%
%
\begin{document}
% Title page frame
\begin{frame}
    \titlepage 
\end{frame}

% Remove logo from the next slides
\logo{}

% Outline frame
\begin{frame}{Sumário}
    \tableofcontents
\end{frame}
%% -------------------------------- %% -----------------------------------%%
%% Section 1 - Introductory concepts
\section{Introdução}
%
%\subsection{Objetivos}
%
\begin{frame}
    \frametitle{Objetivos iniciais}
    \begin{alertblock}{Estudo da lógica proposicional}
        \begin{itemize}
            \item Representar e especificar os conceitos de sintaxe e semântica associados a qualquer lógica utilizada ou linguagem.
            \item Estudar os métodos que produzem ou verifiquem as fórmulas ou argumentos utilizados.
            \item Definir sistemas de dedução formal onde são consideradas as noções de prova e consequência lógica.
            \item Correlacionar diagramas de Venn com a prática.
            \item Conhecer a álgebra de Boole.
        \end{itemize}
    \end{alertblock}
\end{frame}
%
\subsection{Definições iniciais}
%
\begin{frame}
    \frametitle{Definições}
    \framesubtitle{Introductory definitions to the course}
    \textbf{Proposição}\\
    \quad $\star$ \textcolor{blue}{É qualquer conjunto de palavras ou símbolos que expressam um pensamento completo.}\\
    \quad $\star$ As proposições transmitem fatos ou exprimem juízos que formamos a respeito de determinado acontecimento.\\ \pause
    \textbf{Exemplos}
    \begin{itemize}
        \item A lua é um satélite da Terra.\pause
        \item O valor arredondado de $\pi$ vale $3,14$.\pause
        \item Recife é a capital da Paraíba\pause
        \item $\cos(90^o)~=~0$.\pause
    \end{itemize}
    \textbf{Alfabeto}\\
    \quad $\star$ É o conjunto de símbolos usado em qualquer linguagem. A seguir a tabela de símbolos usados na disciplina é apresentado:
\end{frame}
%
\begin{frame}
    \frametitle{Definições iniciais}
    \framesubtitle{Introductory definitions to start the course}
    \begin{block}<1->{Alfabeto da lógica proposicional}
        \begin{itemize}
            \item Símbolo de pontuação: (,)
            \item Símbolos booleanos: \textit{true, false}
            \item Símbolos proposicionais simples: $p,~q,~r,~s,~p_1,~q_2$
            \item Símbolos proposicionais compostos: $P,~Q,~R,~S,~P_1,~Q_1,~S_2$
            \item Conectivos proposicionais: $\land,~\lor,~\lnot,~\rightarrow, ~\leftrightarrow$
        \end{itemize}
    \end{block}
    %
    \begin{block}<2->{Fórmulas}
        \quad São conjuntos de proposições unidos por um conectivo obtendo um valor booleano como resultante. São construídas a partir dos símbolos do alfabeto proposicional.\\
        \quad \textcolor{red}{Tal como ocorre nas linguagens faladas ou escritas, não é qualquer concatenação de símbolos que é uma fórmula.}
    \end{block}
\end{frame}
%
\begin{frame}
    \frametitle{Algumas definições}
    \framesubtitle{Introductory definitions to start the course}
    \begin{itemize}
        \item <3-> Exemplo:
    \end{itemize}

\end{frame}
%
\subsection{Princípios fundamentais da lógica matemática}
%
\begin{frame}
    \frametitle{Princípios da lógica clássica}
    %\framesubtitle{Basic principles compared to fundamental rules to mathematical logic}
    %
    \pause
    \begin{block}{Princípio da identidade}
        Uma proposição não pode ser verdadeira e falsa ao mesmo tempo. \\
        $$P~\text{é igual a}~P$$
    \end{block}
    \pause
    %
    \begin{block}{Princípio da não contradição}
        Uma proposição não pode ser \textit{verdadeira} e \textit{falsa} ao mesmo tempo.
        $$ \text{não}~\left(P~\text{e não}~P\right) $$
    \end{block}
    \pause
    %
    \begin{block}{Princípio do terceiro excluído}
        Toda proposição ou é verdadeira ou é falsa, não existindo um terceiro valor que ela possa assumir.
        $$ P~\text{ou não}~P~(\otimes - \text{ou exclusivo})$$
    \end{block}
\end{frame}
%
\subsection{Tipos de proposições}
%
\begin{frame}
    \frametitle{Proposição simples e compostas}
    \framesubtitle{Simple or compound preposition}
    $\star$ \textbf{Proposições simples}\\
    É aquela que contêm somente uma afirmação.\\
    \textbf{Exemplo:}\\
    \textcolor{teal}{Nós somos ricos.}\\
    \textcolor{teal}{Não como todo dia.}\\
    %
    \hfill \break
    $\star$ \textbf{Proposições compostas}\\
    Uma proposição é dita composta quando for constituída por uma sequência finita de pelo menos duas proposições.\\
    \textbf{Exemplo:}\\
    \textcolor{teal}{Vamos ao cinema ou ao teatro.}\\
    \textcolor{teal}{O céu é azul e cheio de nuvens.}
\end{frame}
%
\subsection{Conectivos proposicionais}
%
\begin{frame}
    \frametitle{Conectivos do cálculo proposicional}
    \framesubtitle{Conectors for all arithmetic with propositions.}
    Na linguagem comum, usam-se palavras explícitas ou não para interligar frases dotadas de algum sentido.Tais palavras são substituídas, na Lógica Matemática, por símbolos denominados conectivos lógicos.
\end{frame}
\begin{frame}{Lists in Beamer}

This is an unordered list:
\begin{itemize}
    \item Item 1
    \item Item 2
    \item Item 3
\end{itemize}

and this is an ordered list:
\begin{enumerate}
    \item Item 1
    \item Item 2
    \item Item 3
\end{enumerate}

\end{frame}
%%
%
%%% -------------------------------- %% -----------------------------------%%
% Section 2 - Fundamental operations under logic propositions
\section{Operações lógicas com proposições}
%
\subsection{Conjunção}
%
\begin{frame}
    \frametitle{Operação de conjunção ($\land$ - 'e' lógico)}
    \framesubtitle{Logical AND operation with propositions}
\end{frame}
%
\subsection{Disjunção}
%
\subsection{Condicional}
%
\subsection{Bicondicional}
%
\subsection{Negação}

\begin{frame}{Blocks in Beamer}
    \begin{block}{Standard Block}
        This is a standard block.
    \end{block}
    \begin{alertblock}{Alert Message}
        This block presents alert message.
    \end{alertblock}
    \begin{exampleblock}{An example of typesetting tool}
        Example: MS Word, \LaTeX{}
    \end{exampleblock}
\end{frame} 

\end{document}