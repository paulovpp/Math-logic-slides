\documentclass[10pt]{beamer}
\usepackage[utf8]{inputenc}
\usepackage[T1]{fontenc}
\usepackage[brazilian]{babel}

\usetheme{Warsaw}
\usecolortheme{default}
\usefonttheme{professionalfonts}
\graphicspath{{./imgs/}}
\setbeamertemplate{caption}[numbered]
\setbeamercovered{transparent}
\geometry{left=1cm, right=1cm}

\title{Lógica Matemática} 
\author[Paulo Pinheiro]
{Dr. Paulo Vinicius Pereira Pinheiro\inst{1}}
\institute[UNIFAP]
{
    \inst{1}
    Centro Universitário Paraíso do Ceará\\
    UNIFAP
}
%
\date{Última atualização:\\ \today}
\logo{\large \LaTeX{}}
\logo{\includegraphics[height=0.8cm]{imgs/UNIFAP.png}}
%%%%%%%%%%%%%%%%%%%%%%%%%%%%%%%%%%%%%%%%%%%%%%%%%%%%%%%%%%%%%%%%%%%%%%%%%%%%%
\begin{document}
% Title page frame
\begin{frame}
    \titlepage 
\end{frame}

% Remove logo from the next slides
\logo{}

% Outline frame
\begin{frame}{Sumário}
    \tableofcontents
\end{frame}
%% -------------------------------- %% -----------------------------------%%
%% Lists of frames
\section{Introdução}
\subsection{Conceitos fundamentais}
%
\subsubsection{Objetivos iniciais}
%
\begin{frame}
    \frametitle{Objetivos iniciais}
    \begin{alertblock}{Estudo da lógica proposicional}
        \begin{itemize}
            \item Representar e especificar os conceitos de sintaxe e semântica associados a qualquer lógica utilizada ou linguagem.
            \item Estudar os métodos que produzem ou verifiquem as fórmulas ou argumentos utilizados.
            \item Definir sistemas de dedução formal onde são consideradas as noções de prova e consequência lógica.
            \item Correlacionar diagramas de Venn com a prática.
            \item Conhecer a álgebra de Boole.
        \end{itemize}
    \end{alertblock}
\end{frame}
%
\subsubsection{Definições}
%
\begin{frame}
    \frametitle{Definições}
    \framesubtitle{Introductory definitions to start the course}
    \textbf{Proposição}\\
    \quad $\star$ \textcolor{red}{É qualquer conjunto de palavras ou símbolos que expressam um pensamento completo.}\\
    \quad $\star$ As proposições transmitem fatos ou exprimem juízos que formamos a respeito de determinado acontecimento.\\ \pause
    \textbf{Exemplos}
    \begin{itemize}
        \item A lua é um satélite da Terra.\pause
        \item O valor arredondado de $\pi$ vale $3,14$.\pause
        \item Recife é a capital da Paraíba\pause
        \item $\cos(90^o)~=~0$.\pause
    \end{itemize}
    \textbf{Alfabeto}\\
    \quad $\star$ É o conjunto de símbolos usado em qualquer linguagem. A seguir a tabela de símbolos usados na disciplina é apresentado:
\end{frame}
%
\begin{frame}
    \frametitle{Definições}
    \framesubtitle{Introductory definitions to start the course}
    \begin{block}{Alfabeto da lógica proposicional}
        \begin{itemize}
            \item Símbolo de pontuação: (,)
            \item Símbolos booleanos: \textit{true, false}
            \item Símbolos proposicionais simples: $p,~q,~r,~s,~p_1,~q_2$
            \item Símbolos proposicionais compostos: $P,~Q,~R,~S,~P_1,~Q_1,~S_2$
            \item Conectivos proposicionais: $\land,~\lor,~\lnot,~\rightarrow, ~\leftrightarrow$
        \end{itemize}
    \end{block}
\end{frame}
%
\begin{frame}{Lists in Beamer}

This is an unordered list:
\begin{itemize}
    \item Item 1
    \item Item 2
    \item Item 3
\end{itemize}

and this is an ordered list:
\begin{enumerate}
    \item Item 1
    \item Item 2
    \item Item 3
\end{enumerate}

\end{frame}


% Blocks frame
\section{Blocks in Beamer}
\begin{frame}{Blocks in Beamer}
    \begin{block}{Standard Block}
        This is a standard block.
    \end{block}
    \begin{alertblock}{Alert Message}
        This block presents alert message.
    \end{alertblock}
    \begin{exampleblock}{An example of typesetting tool}
        Example: MS Word, \LaTeX{}
    \end{exampleblock}
\end{frame} 

\end{document}