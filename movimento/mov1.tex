\documentclass{beamer}
\usepackage[utf8]{inputenc}
\usepackage[T1]{fontenc}
\usepackage[brazilian]{babel}
\usepackage{graphicx}
\usepackage{caption}

\graphicspath{{./imgs/}}
\setbeamertemplate{caption}[numbered]
\setbeamercovered{transparent}

\usetheme{Warsaw}
\usecolortheme{default}

\title{Movimento e dinâmica} 
\author[Paulo Pinheiro]
{Dr. Paulo Pinheiro\inst{1}}
\institute[UNIFAP]
{
    \inst{1}
    Centro Universitário Paraíso do Ceará\\
    UNIFAP
}

\date{Última atualização:\\ \today}
\logo{\large \LaTeX{}}
\logo{\includegraphics[height=0.8cm]{UNIFAP.png}}
%
\begin{document}
% Title page frame
\begin{frame}
    \titlepage 
\end{frame}
%
% Remove logo from the next slides
\logo{}
% Outline frame
\begin{frame}{Outline}
    \tableofcontents
\end{frame}
%--------------------  FRAMES  ------------------------
%%
% Lists frame
%%
\section{Introdução}
%
\subsection{Preparação}
%
\begin{frame}{Formação inicial}
    \begin{columns}[T] % align columns
        \begin{column}{.7\textwidth}
            \textbf{Formação Acadêmica}
            \begin{itemize}
                \item<1-> Ms. e Dr. em Eng. de Teleinformática
                \item<1-> Bacharel em Física
                \item<1-> Graduando em Eng. da Computação
            \end{itemize}
            %
            \vspace{0.2cm}
            %
            %\pause
            \textbf{Atuação Acadêmica}
            \begin{itemize}
                \item<2-> Professor de ensino superior de cursos de Engenharia e TI
                \item<2-> Instrutor de cursos técnicos nas áreas de TI e elétrica
                \item<2-> Gerente de projetos e inovação
            \end{itemize}
        \end{column}%
        %
        \hfill%
        %
        \begin{column}{.35\textwidth}
            
            
        \end{column}%
    \end{columns}
\end{frame}
%
\subsection{Conceitos iniciais}
%%
\section{Conceitos iniciais}
%
\subsection{Grandezas}
%
\subsection{Análise dimensional}
%
\subsection{Notação científica}
%
\subsection{Múltiplos e submúltiplos}
%%
\section{Movimento}

\end{document}